\section{Conclusion and Future Work}

We presented three balancing methods and two vertical clustering combination approaches to address the load balance problem when clustering workfl�ow tasks. We also defi�ned three imbalance metrics to quantitatively measure workfl�ow characteristics based on task runtime variation (HRV), task impact factor (HIFV), and task distance variance (HDV).

Three experiment sets were conducted using traces from five real workflow applications. The first experiment aimed at measuring the performance gain over a control execution without clustering. In addition, we compared our balancing methods with three algorithms from the literature. Experimental results show that our methods yields significant improvement over a control execution, and that they have acceptable performance when compared to the best estimated performance of the existing algorithms. The second experiment measured the influence of average data size and number of available resources on the performance gain. Particularly, results show that our methods have different sensitivity to data- and computational-intensive workflows. Finally, the last experiment evaluated the interest of performing horizontal and vertical clustering in the same execution. Results show that vertical clustering can significantly improve pipeline-structured workflows, but it may decrease the performance if the workflow has no explicit pipelines.

The simulation based evaluation also shows that the performance improvement of the proposed balancing algorithms (HRB, HDB and HIFB) are highly related to the metric values (HRV, HDV and HIFV) that we introduced. For example, a workflow with high HRV tends to have better performance improvement with HRB since HRB is used to balance the runtime variance. 

In the future, we plan to further analyze the imbalance metrics proposed. For instance, the values of these metrics presented in this work are not normalized, and thus their values per level (HIFV, HDV, and HRV) are in different scales. Also, we plan to analyze more workflow applications, particularly the ones with asymmetric structures, to investigate the relationship between workflow structures and the metric values. 

Also, as shown in Figure~\ref{fig:evaluation_vc_genome}, \emph{VC-prior} can generate very large clustered jobs vertically and makes it difficult to horizontal methods to improve further. Therefore, we aim to develop imbalance metrics for \emph{VC-prior} to avoid generating large clustered jobs, i.e., based on the cumulated runtime of tasks in a pipeline. 


As shown in our experiment results, the combination of our balancing methods with vertical clustering have different sensitivity to workflows with distinguished graph structures and runtime distribution. Therefore, a possible future work is the development of a portfolio scheduling, which chooses multiple scheduling algorithms, and dynamically selects most suitable one according to the dynamic load.

In this paper, we illustrate the performance gain of combining horizontal clustering methods and vertical clustering. We plan to combine multiple algorithms together instead of just two. We will develop policy engine that iteratively choose one algorithm from all of the balancing methods based on the imbalance metrics until the performance gain converges. 

Last, we aim at applying our metrics to other workflow study areas, such as workflow scheduling where heuristics would either look into the characteristics of the task when it is ready to schedule (local scheduling), or examine the entire workflow (global optimization algorithms). In this work, the impact factor metric only uses a family of tasks that are tightly related or similar to each other. This method represents a new approach to solve the existing problems. 

\section{Acknowledgements}
This work is funded by NSF IIS-0905032 and NSF FutureGrid 0910812. We thank Gideon Juve, Karan Vahi, Rajiv Mayani, and Mats Rynge for their help. 