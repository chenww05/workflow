% Section
\section{Related Work}
\label{sec:related-work}

Overhead analysis~\cite{Ostberg2011, Prodan2008} is a topic of great interest in the Grid community. Stratan et al.~\cite{Stratan2008} evaluate in a real-world environment Grid workflow engines including DAGMan/Condor and Karajan/Globus. Their methodology focuses on five system characteristics: overhead, raw performance, stability, scalability, and reliability. They pointed out that head node consumption should not be negligible and the main bottleneck in a busy system is often the head node. Prodan et al.~\cite{Prodan2008} offered a complete Grid workflow overhead classification and a systematic measurement of overheads. In Chen et al.~\cite{Chen2011}, we extended~\cite{Prodan2008} by providing a measurement of major overheads imposed by workflow management systems and execution environments and analyzed how existing optimization techniques improve runtime by reducing or overlapping overheads. The prevalent existence of system overheads is an important reason why task clustering provides significant performance improvement for workflow-based applications. In this paper, we aim to further improve the performance of task clustering under imbalanced load. 

The low performance of \emph{fine-grained} tasks is a common problem in widely distributed platforms where the scheduling overhead and queuing times at resources are high, such as Grid and Cloud systems. Several works have addressed the control of task granularity of bags of tasks. For instance, Muthuvelu et al.~\cite{Muthuvelu:2005:DJG:1082290.1082297} proposed a clustering algorithm that groups bags of tasks based on their runtime---tasks are grouped up to the resource capacity. Later, they extended their work~\cite{4493929} to determine task granularity based on task file size, CPU time, and resource constraints. Recently, they proposed an online scheduling algorithm~\cite{Muthuvelu2010,Muthuvelu2013170} that groups tasks based on resource network utilization, user's budget, and application deadline. Ng et al.~\cite{keat-2006} and Ang et al.~\cite{ang-2009} introduced bandwidth in the scheduling framework to enhance the performance of task scheduling. Longer tasks are assigned to resources with better bandwidth. Liu and Liao~\cite{Liu2009} proposed an adaptive fine-grained job scheduling algorithm to group fine-grained tasks according to processing capacity and bandwidth of the current available resources. Although these techniques significantly reduce the impact of scheduling and queuing time overhead, they did not consider data dependencies.

Task granularity control has also been addressed in scientific workflows. For instance, Singh et al.~\cite{Singh:2008:WTC:1341811.1341822} proposed a level- and label-based clustering. In level-based clustering, tasks at the same level can be clustered together. The number of clusters or tasks per cluster are specified by the user. In the label-based clustering, the user labels tasks that should be clustered together. Although their work considers data dependencies between workflow levels, it is done manually by the users, which is prone to errors. Recently, Ferreira da Silva et al.~\cite{Ferreira-granularity-2013} proposed task grouping and ungrouping algorithms to control workflow task granularity in a non-clairvoyant and online context, where none or few characteristics about the application or resources are known in advance. Their work significantly reduced scheduling and queuing time overheads, but did not consider data dependencies.

A plethora of balanced scheduling algorithms have been developed in the networking and operating system domains. Many of these schedulers have been extended to the hierarchical setting. Lifflander et al.~\cite{Lifflander} proposed to use work stealing and a hierarchical persistence-based rebalancing algorithm to address the imbalance problem in scheduling. Zheng et al.~\cite{Zheng} presented an automatic hierarchical load balancing method that overcomes the scalability challenges of centralized schemes and poor solutions of traditional distributed schemes. There are other scheduling algorithms~\cite{Braun2001} (e.g. list scheduling) that indirectly achieve load balancing of workflows through makespan minimization. However, the benefit that can be achieved through traditional scheduling optimization is limited by its complexity. The performance gain of task clustering is primarily determined by the ratio between system overheads and task runtime, which is more substantial in modern distributed systems such as Clouds and Grids. 

Workflow patterns~\cite{Yu2005, Juve2013, Liu2008} are used to capture and abstract the common structure within a workflow and they give insights on designing new workflows and optimization methods. Yu and Buyya~\cite{Yu2005} proposed a taxonomy that characterizes and classifies various approaches for building and executing workflows on Grids. They also provided a survey of several representative Grid workflow systems developed by various projects world-wide to demonstrate the comprehensiveness of the taxonomy. Juve et al.~\cite{Juve2013} provided a characterization of workflow from 6 scientific applications and obtained task-level performance metrics (I/O, CPU, and memory consumption). They also presented an execution profile for each workflow running at a typical scale and managed by the Pegasus workflow management system~\cite{Deelman:2005:PFM:1239649.1239653}. Liu et al.~\cite{Liu2008} proposed a novel pattern based time-series forecasting strategy which utilizes a periodical sampling plan to build representative duration series. We illustrate the relationship between the workflow patterns (asymmetric or symmetric workflows) and the performance of our balancing algorithms. 

Some work in the literature has further attempted to define and model workflow characteristics with quantitative metrics. In~\cite{Ali2004}, the authors proposed a robustness metric for resource allocation in parallel and distributed systems and accordingly customized the definition of robustness. Tolosana et al.~\cite{Tolosana2011} defined a metric called Quality of Resilience to assess how resilient workflow enactment is likely to be in the presence of failures. Ma et al. ~\cite{Ma:2014:GDB:2560969.2561388} proposed a graph distance based metric for measuring the similarity between data oriented workflows with variable time constraints, where a formal structure called time dependency graph (TDG) is proposed and further used as representation model of workflows. Similarity comparison between two workflows can be reduced to computing the similarity between TDGs. In this paper, we focus on novel quantitative metrics that are able to demonstrate the imbalance problem in scientific workflows. 

