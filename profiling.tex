
\section{Overlapping Metrics}
\label{sec:profiling}

The realistic characteristics of workflows are critical to optimal workflow orchestration and profiling is an effective approach to investigate the behaviors of many complex applications. In this paper, we particularly focus on the overlapped and cumulative overheads of workflow events because this approach represents a promising trend of optimizing workflows and there is a lack of supporting analysis tools.

Workflow is typically a graph of computational activities, data transfer and system overheads as shown in our o-DAG model. Different branches of these activities may overlap with each other in the timeline and our major concern is their cumulative projection on the timeline or makespan. People have been exploring on new approaches to overlap these overheads without the necessity of reducing them. However, the challenge is how these overlapped activities contribute to the eventual makespan and how to reveal their overlapping effects in timeline from different aspects.  

%With cumulative overhead metrics, we can tell whether a workflow optimization method fully utilizes the overlap between overheads and computational activities. 

Fig.~\ref{fig:model_overhead_timeline} shows the timeline of the workflow in Fig.~\ref{fig:model_odag}. In this simple example, we are interested in questions such as: (1) whether the workflow engine delay has a good overlapping within itself? (2) wether the queue delay has a good overlapping with other types of overheads? (3) whether those 'bottleneck' overheads have overlapped with others? In the rest of this section, we will use this workflow as an example to show how we implement these metrics. 

\begin{figure*}[!htb]
	\centering
    \includegraphics[width=0.7\textwidth]{figures/profiling/overhead_timeline.pdf}
    \caption{Workflow Timeline}
    \label{fig:profiling_overhead_timeline}
\end{figure*}

\begin{figure*}[!htb]
	\centering
 \includegraphics[width=0.8\textwidth]{figures/profiling/rr.pdf}
    \caption{Impact Factor}
    \label{fig:profiling_overhead_rr}
\end{figure*}


\subsection{Metrics to Evaluate Cumulative Overheads and Runtimes}

In this section, we propose four metrics to calculate cumulative overheads of workflows, which are $Sum$, $Projection(PJ)$, $Exclusive~Projection(EP)$ and $Impact~Factor(IF)$. $Sum$ simply adds up the overheads of all jobs without considering their overlap. $PJ$ subtracts from $Sum$ all overlaps of the same type of overhead. It is equal to the projection of all overheads to the timeline. $EP$ subtracts the overlap of all types of overheads from $PJ$. It is equal to the projection of overheads of a particular type excluding all other types of overheads to the timeline. $IF$ uses a reverse ranking algorithm to index overheads and then calculates the cumulative overhead weighted by the ranks (impact factors). The idea is brought by web page indexing algorithms such as PageRank \cite{PageRank1999}. The calculation of $Sum$, $PJ$ and $EP$ is straightforward and Fig.~\ref{fig:profiling_overhead_rr} shows how to calculate the impact factor $(IF)$ of the same workflow graph in Fig.~\ref{fig:profiling_overhead_timeline}. Both $PJ$ and $EP$ are proposed to answer Questions (1) and (2) and $IF$  answers Question (3), while $Sum$ serves as a comparison for them. 

\begin{equation} \label{eq:profiling_rr}
IF(j_u)=d+(1-d)\times\sum_{j_v\in Child(j_u)}{}\frac{IF(j_v)}{||Parent(j_v)||}
\end{equation}

Equation~\ref{eq:profiling_rr} means that the $IF$ of a node (overhead or job runtime) is determined by the $IF$ of its child nodes. $d$ is the damping factor, which controls the speed of the decrease of the $IF$. $||Parent(j_v)||$ is the number of parents that node $j_v$ has. Intuitively speaking, a node is more important if it has more children and/or its children are more important. In terms of workflows, it means an overhead has more power to control the release of other overheads and computational activities. There are two differences compared to the original PageRank:  
\begin{enumerate}
\item We use output link pointing to child nodes while PageRank uses input link from parent nodes, which is why we call it reverse ranking algorithm.
\item Since a workflow is a DAG, we do not need to calculate $IF$ iteratively. For simplicity, we assign the $IF$ of the root node to be 1. And then we calculate the $IF$ of a workflow ($G$) based on the equation below, while $\phi_{j_u}$ indicates the duration of node $j_u$.  

\begin{equation} \label{eq:profiling_sum_rr}
IF(G)=\sum_{}{}IF(j_u) \times \phi_{j_u}
\end{equation}

\end{enumerate}

$IF$ has two important features: (1) $IF$ of a node indicates the importance of this node in the whole workflow graph and its likelihood to be a bottleneck, the greater the more important; (2) $IF$ decreases from the top of the workflow to down, the speed of which is controlled by the damping factor $d$. It reflects the intuition that the overhead occurred at the beginning of the workflow execution is likely to be more important than that at the end.
By analyzing these four types of cumulative overheads, researchers have a clearer view of whether their optimization methods have overlapped the overheads of a same type (if $PJ < Sum$) or other types (if $EP < PJ$). Besides importance, $IF$ is also able to show the connectivity within the workflow, the larger the denser. 
We use a simple example workflow with four jobs to show how to calculate the overlap and cumulative overheads. Fig.~\ref{fig:profiling_overhead_timeline} shows the timeline of our example workflow. $j_1$ and $j_4$ are the parent and children of $j_2$ and $j_3$ respectively.
\begin{table*}[htb!]
\caption{Overhead and Runtime Information}
\label{tab:profiling_stats}
\centering
\begin{tabular}{lrrrrrr}
\hline
Job & Release Time &Queue Delay & Workflow Engine Delay & Runtime &Postscript Delay\\

\hline

$j_1$ & 0 & 10 & 10 & 10 & 10\\ 
$j_2$ & 40 & 10 &20 &50&20\\
$j_3$ &40&10&10&30&10\\
$j_4$ &150&10&10&10&10\\

\hline
\end{tabular}
\end{table*} 


At $t$=0,  $j_1$ is first released and once it is done at $t$=40, $j_3$ and $j_2$ are released. Finally, at $t$=140, $j_4$ is released. Table.~\ref{tab:profiling_stats} shows their overhead and runtime duration (assuming). 
For example, $PJ$ of queue delay is 40 since the queue delay of $j_2$ and $j_3$ overlaps from $t$=50 to 60. $EP$ of queue delay is 10 less than $PJ$ of queue delay since the queue delay of $j_2$ overlaps with the runtime of $j_3$ from $t$=60 to 70. 
%We show how to calculate the cumulative overheads:
 
%For $Sum$:
%$Sum(runtime)=50+30=80$. It contains the time slots of [60, 90] and [70, 120]. 
%$Sum(queue~delay)=10+20+10=40$. It contains [10, 20], [50, 70] and [50, 60]. 
%$Sum(workflow~engine~delay)=10+10+10=30$. It contains [0,10], [40, 50] and [40, 50]. 
%$Sum(postscript~delay)=10+20+10=40$. It contains [30, 40], [90, 100] and [120, 140]. 
%$Sum(data~transfer~delay)=10$. It contains [20, 30].

%For $PJ$:
%$PJ(runtime)=50+30-20=60$. It contains [60, 120].
%$PJ(queue~delay)=10+20+10-10=30$. It contains [10, 20] and [50, 70].
%$PJ(workflow~engine~delay)=10+10+10-10=20$. It contains [0, 10] and [40, 50].
%$PJ(postscript~delay)=10+20+10=40$. It contains [30, 40], [90, 100] and [120, 140].
%$PJ(data~transfer~delay)=10$. It contains [20, 30].

%For $EP$: 
%$EP(runtime)=50+30-20-10-10=40$. It contains [70, 90] and [100, 120].
%$EP(queue~delay)=10+20+10-10-10=20$. It contains [10, 20] and [50, 60]. 
%$EP(workflow~engine~delay)=10+10+10-10=20$. It contains [0, 10] and [40, 50]. 
%$EP(postscript~delay)=10+20+10-10=30$. It contains [30, 40] and  [120,140]. 
%$EP(data~transfer~delay)=10$. It contains [20, 30].

%$RR(runtime)=50\times 0.31+30\times 0.31=24.8$.
%$RR(queue~delay)=10\times 1.00+10\times 0.41+10\times 0.41=18.2$.
%$RR(workflow~engine~delay)=10\times 1.00+10\times 0.50+10\times 0.50=20$.
%$RR(postscript~delay)=10\times 1.00+10\times 0.19+20\times 0.19=15.7$.
%$RR(data~transfer~delay)=10\times 1.00=10$.

For simplicity, we do not include the data transfer delay, prescript delay and others. The overall makespan for this example workflow is 180. Table~\ref{tab:model_percentage_overhead} shows the percentage of overheads and job runtime over makespan.  

\begin{table}[h!]
\caption{Percentage of Overheads and Runtime}
\label{tab:model_percentage_overhead}
\centering
\begin{tabular}{lrrrr}
\hline
Contribution & Sum & PJ & EP &IF\\

\hline

runtime(\%) & 57.1 & 42.8 & 28.5 &17.7 \\
queue delay(\%) & 28.5 &21.4 &14.2 &13.0 \\
workflow engine delay(\%) & 21.4 &14.2& 14.2 &14.2\\
postscript delay(\%) & 28.5 & 28.5 & 21.4 & 11.2 \\
\hline
\end{tabular}
\end{table} 


In Table~\ref{tab:model_percentage_overhead}, we can conclude that the sum of $Sum$ is larger than makespan and smaller than makespan$\times$(number of resources) because it does not count the overlap at all. $PJ$ is larger than makespan since the overlap between more than two types of overheads may be counted twice or more. $EP$ is smaller than makespan since some overlap between more than two types of overheads may not be counted.  $RR$ shows how intensively these overheads and computational activities are connected to each other. 

\subsection{Experiments and Evaluations}

We applied these four metrics to a widely used workflow in our experiments. These workflows were run on distributed platforms including clouds, grids and dedicated clusters. 
On clouds, virtual machines were provisioned and then the required services (such as file transfer services) were deployed. 
We examined two clouds: Amazon EC2 \cite{AmazonEC2}  and FutureGrid \cite{FutureGrid}. Amazon EC2 is a commercial, public cloud that is been widely used in distributed computing. 
We examined the overhead distributions of a widely used astronomy workflow called Montage \cite{Berriman2004} that is used to construct large image mosaics of the sky. Montage was run on FutureGrid \cite{FutureGrid}. FutureGrid is a distributed, high-performance testbed that provides scientists with a set of computing resources to develop parallel, grid, and cloud applications. 
%Should be included in final defense
\subsection{Relationship between Overhead Metrics and Overall Performance}

In this section, we aim to investigate the relationship between the overhead metrics that we proposed and the overall performance of popular workflow restructuring techniques. Among them, task clustering \cite{Singh2008} is a technique that increases the computational granularity of tasks by merging small tasks together into a clustered job, reducing the impact of the queue wait time and also the makespan of the workflow. Data or job throttling \cite{Humphrey2008} limits the amount of parallel data transfer to avoid overloading supporting services such as data servers. Throttling is especially useful for unbalanced workflows in which one task might be idle while waiting for data to arrive. The aim of throttling is to appropriately regulate the rate of data transfers between the workflow tasks via data transfer servers by ways of restricting the data connections, data threads or data transfer jobs. Provisioning tools often deploy pilot jobs as placeholders for the execution of application jobs. Since a placeholder can allow multiple application jobs to execute during its lifetime, some job scheduling overheads can be reduced. 

\textbf{How Task Clustering Reduces Overheads}

In the following sections, we use a Montage workflow to show how different optimization methods improve overall performance. Many workflows are composed of thousands of fine computational granularity tasks. Task clustering is a technique that increases the computational granularity of tasks by merging small jobs together into a clustered job, reducing the impact of the queue wait time and minimizing the makespan of the workflow. Table 4.2 compares the overheads and runtime of the Montage workflow. We can conclude that with clustering, although the average overheads do not change much, the cumulative overheads decrease greatly due to the decreased number of jobs. With clustering, the makespan has been reduced by 53.3\% by reducing the number of all jobs from 3461 to 104 in this example. Fig. 4.5 shows the percentage of workflow overheads and runtime. The percentage is calculated by the cumulative overhead ($PJ$, or $EP$) divided by the makespan of workflows. With clustering, the portion of runtime is increased significantly. Fig. 4.6 profiles the number of active jobs during execution and it also shows that with clustering the resource utilization is improved significantly. 

\textbf{How Job Throttling Reduces Overheads}

Data or job throttling [13] limits the amount of parallel data transfer to avoid overloading supporting services such as data servers. Throttling is especially useful for unbalanced workflows in which one task might be idle while waiting for data to arrive. The aim of throttling is to appropriately regulate the rate of data transfers between the workflow tasks via data transfer servers by ways of restricting the data connections, data threads or data transfer jobs. Especially on cloud platforms, I/O requests need to go through more layers than a physical cluster; and thereby workflows may suffer a higher overhead from data servers.

In our experiments, the data transfer service is deployed on a virtual machine that is similar to a worker node.  In this section, we evaluate a simple static throttling strategy where the Condor scheduler limits the number of concurrent jobs to be run and thereby restricts the number of parallel I/O requests. There are 32 resources available and we evaluate the cases with throttling parameters that are equal to 24, 16 and 12 in Table 4.3. In the case of 24, the resources are better utilized but the data server is heavily loaded. In the case of 12, the resources are under-utilized even though the data server has more capabilities. In this experiment, both $PJ$ and $EP$ reflect the variation trend of overheads and makespan better than $Sum$. 

Fig. 4.7 shows the percentage of workflow overheads and runtime. Fig. 4.8 profiles the number of active jobs during execution. Montage is an unbalanced workflow because the three major types of jobs (mProjectPP, mDiffFit, and mBackground) impose a heavy load on the data server while the other jobs in the workflow do not. Fig. 4.8 shows that with throttling the maximum number of active jobs is restricted. With limited throttling (reducing threshold from 24 to 16), the data transfer requests are distributed in the timeline more evenly and, as a result, their overhead is reduced. However, with over throttling (reducing threshold from 16 to 12), resources are not fully utilized and thus the makespan is increased. 

\textbf{How Pre-staging Reduces Overheads}

Scientific workflows often consume and produce a large amount of data during execution. Data pre-staging [14] transfers input data before the computational activities are started or even before the workflow is mapped onto resources. Data placement policies distribute data in advance by placing data sets where they may be requested or by replicating data sets to improve runtime performance. In our experiments, because data is already pre-staged, the implementation of the stage-in job is to create a soft link to the data from the workflow’s working directory, making it available to the workflow jobs. Table 4.4 and Fig. 4.9 show the cumulative overheads and runtime of the Montage workflows running with and without pre-staging. Looking at the rows for $PJ$ in Table 4.4, we can conclude that pre-staging improves the overall runtime by reducing the data transfer delay. For the case without pre-staging the $EP$ for data transfer delay is zero because it overlaps with the workflow engine delay of another job. Therefore, in this experiment, $PJ$ reflects the variation trend of the makespan more consistently. 

\textbf{How Provisioning Reduces Overheads}

Many of the scientific applications presented here consist of a large number of short-duration tasks whose runtimes are greatly influenced by overheads present in distributed environments. Most of these environments have an execution mode based on batch scheduling where jobs are held in a queue until resources become available to execute them. Such a best-effort model normally imposes heavy overheads in scheduling and queuing. For example, Condor-G [23] uses Globus GRAM [37] to submit jobs to remote clusters. The Globus Toolkit normally has a significant overhead compared to running Condor directly as an intra domain resource and job management system. Provisioning tools often deploy pilot jobs as placeholders for the execution of application jobs. Since a placeholder can allow multiple application jobs to execute during its lifetime, some job scheduling overheads can be reduced. In our experiments, we compared the performance of Condor-G (without provisioning) and Corral (with provisioning). 

Table 4.5 and Fig. 4.10 show the percentage of workflow overheads and runtime. The percentage is calculated by the cumulative overhead ($Sum$, $PJ$, or $EP$) divided by the makespan of workflows. Comparing $Sum$, $PJ$ and $EP$, we can conclude that the overheads with provisioning have been reduced significantly because the local scheduler has direct control over the resources without going through Globus. 


